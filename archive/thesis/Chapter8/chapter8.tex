\chapter{Conclusions}
\label{chapter 8}

\ifpdf
    \graphicspath{{Chapter8/Figs/}{Chapter8/Figs/PDF/}{Chapter8/Figs/}}
\else
    \graphicspath{{Chapter8/Figs/Vector/}{Chapter8/Figs/}}
\fi
\section{Summary}
Industry 4.0 paradigm is shaping modern manufacturing systems, leading to the implementation of smart production lines and to the design of Digital Twins, that are dynamic software copies of real systems.\\
In this thesis the issue of a Digital Twin alignment with a real system is addressed, proposing a procedure to update the process model which the Digital Twin of a production line is based on. This method is composed of three steps, that are Change Detection, Change Identification, and Model Modification: the first step consists in figuring out if a process variation verified, by monitoring KPIs extracted from production data in the form of event logs. In the second step, the KPIs are analyzed and classified to define what kind of variation occurred. In the third step, the model representing the system is modified accordingly to the identified variation. The approach novelty lies in the analysis (i.e. detection and identification) of a system variation before the actual model modification: in this way a precise and local update of the model is possible, reducing the effort to keep the Digital Twin aligned and making it a more reliable and less noise-affected copy of the system. The dissertation discusses the first and second step of the method, leaving the third step as subject of future studies.\\
The thesis is devised as part of the Process Mining field of research, since it aims to exploit event logs, generated by sensors embedded in production lines, as data sources for the computation of KPIs. It has been decided to study the KPI behaviors considering some specific system structures and process variations: the production lines are assumed to be (initially) stable and unsaturated Simple Serial Lines, composed of limited buffer stages having negligible transfer time among them, with stochastic job arrivals and processing time in machines, no resource breakdowns, blocking after service policy, and unconstrained departures; the considered systems are sensorized, having for each stage three sensors that register in an event log the instant in which a case (i.e. a job) passes through; finally, the variations assumed to occur during the process are permanent and sudden changes of the resource processing capacities and of the buffer capacities. \\
The KPI extractions from the event log are described, providing the formulas to compute them and explaining what process aspect they represent. Then, the KPI behaviors are displayed and analyzed through a series of simulations imitating production lines having the characteristics described above. Finally, these behaviors are collected to create a map of variation, that is a set of schemes helping to relate the KPI changes with the underlying real system variations. \\
This thesis showed that it is possible to detect and, even more important, to understand what kind of variation is occurring in a system by monitoring and analyzing KPIs extracted from event logs. 
\section{Future developments}
As stated in chapter \ref{chapter 2}, this work aimed to lay the foundations of a new process model update procedure. Many research paths are left unexplored, first and foremost the design of an algorithm to automatically detect and identify process variations, and then addressing the issue of model partial modifications. Moreover, the assumptions made concerning the system should be relaxed, getting to study more complex production line structures (e.g. assembly lines, multi-product systems, occurrence of breakdowns) and different types of variations (e.g. system structure variations, incremental changes). The influence of length and shift parameters on Rolling Windows KPIs should be examined to find a balance between accuracy and detection speed: indeed, wider and overlapping windows better filter data from noise and outliers; on the other side, narrower and non-contiguous windows are more reactive to changes. Finally, other KPI metrics should be considered, computing not only the basic KPIs average on rolling windows, but also moments of higher orders have to be analyzed to deepen the knowledge of the indicator behaviors.